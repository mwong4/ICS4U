%%
%%
\documentclass[12pt]{book}
\usepackage[utf8]{inputenc}
\usepackage{amsfonts,amssymb,amsmath,amsthm}
\usepackage{graphicx}
\usepackage{hyperref}
\usepackage{boxedminipage}
\usepackage{lastpage}
\usepackage{graphicx}
\usepackage{caption}
\usepackage{setspace}
\usepackage{polynom}
\usepackage{hyperref}
\usepackage{array}
\usepackage{geometry}
\usepackage{bm} %for bold math font
\newcolumntype{C}[1]{>{\centering\let\newline\\\arraybackslash\hspace{0pt}}m{#1}}
\setlength{\textheight}{10in}
\setlength{\textwidth}{7.4in}
\setlength{\topmargin}{-0.75in}
\setlength{\oddsidemargin}{-0.5in}
\setlength{\evensidemargin}{-0.5in}
\setlength{\parskip}{0.15in}
\setlength{\parindent}{0in}

%commands
\newcommand\numberthis{\addtocounter{equation}{1}\tag{\theequation}}

\begin{document}

\vspace{-1.0in}\begin{center}
\Large{Advanced Functions }

\Large{Assignment \#1}


\end{center}

%\medskip

\vspace{0.015in}\hrulefill\ 

\textbf{Reference Declaration} %  Fill in your Reference Declarations in this section before your submit your assignment.

Complete the Reference Declaration section below in order for your assigment to be graded.

If you used any references beyond the course text and class notes (such as other texts, discussions with peers or online resources), indicate this information in the space below.  If you did not use any aids then explicitly state this in the space provided. 

Be sure to cite appropriate theorems throughout your work. You may use shorthand for well-known theorems like the FT (Factor Theorem), RRT (Rational Root Theorem), etc. 

Note: Your submitted work must be \textbf{your original work}. 

Family Name: Wong%Family Name Here
First Name: Max%First Name Here

\textbf{Declared References:} 

Used this forum answer to number the an element within align (Question 2): \\
\href{https://tex.stackexchange.com/questions/42726/align-but-show-one-equation-number-at-the-end}{stackexchange align* but show one equation number at the end}

% Type your references here.
% You can use as many lines as required.

\vspace{0.015in}\hrulefill\ 

\newpage


% INSTRUCTIONS SECTION
\section*{Instructions}
\begin{center}
\setlength{\fboxrule}{2pt}
\begin{boxedminipage}{6.5in}
1.	Organize and express complete, effective and concise responses to each problem.\\
2.	Use appropriate mathematical conventions and notation wherever possible.\\
3.	Provide logical reasoning for your arguments and cite any relevant theorems. \\
4.  Ask your teacher questions if you need any clarification.
\end{boxedminipage}
\end{center} 

% EVALUATION SECTION
\section*{Evaluation}

% LEARNING EXPECTATION(S)
\begin{itemize}
\item[D3]	Students will compare the characteristics of functions, and solve problems by modelling and reasoning with functions.
\end{itemize}

% RUBRIC
\begin{tabular}{| C{2in} | C{1in} | C{1in} | C{1in} | C{1in} |}
\hline
\textbf{Criteria} & \textbf{Level 1} & \textbf{Level 2} & \textbf{Level 3} & \textbf{Level 4} \\
\hline
\emph{Understanding of Mathematical Concepts} & Demonstrates limited understanding & Demonstrates some understanding & Demonstrates considerable understanding & Demonstrates thorough understanding of concepts \\
\hline
\emph{Selecting Tools and Strategies} & Selects and applies appropriate tools and strategies, with major errors, omissions, or mis-sequencing & Selects and applies appropriate tools and strategies, with minor errors, omissions, or mis-sequencing & Selects and applies appropriate tools and strategies accurately, and in a logical sequence & Selects and applies appropriate and efficient tools and strategies accurately to create mathematically elegant solutions \\
\hline
\emph{Reasoning and Proving} & Inconsistently or erroneously employs logic to develop and defend statements & Statements are developed and defended with some omissions or leaps in logic & Frequently develops and defends statements with reasonable logical justification & Consistently develops and defends statements with sophisticated and/or complete logical justification \\
\hline
\emph{Communicating} & Expresses and organizes mathematical thinking with limited effectiveness & Expresses and organizes mathematical thinking with some effectiveness & Expresses and organizes mathematical thinking with considerable effectiveness & Expresses and organizes mathematical thinking with a high degree of effectiveness \\
\hline
\end{tabular}

\pagebreak



%%%%%%%%%%%% PROBLEMS START HERE

\begin{enumerate}

%% PROBLEM 1
\item  \textbf{Describe} the characteristics of the function $f(x) = -2|x-3|+2$ by filling in the table given below. \textbf{Write} a paragraph briefly explaining how you determined each characteristic.\\

\vspace{0.3 cm}

\renewcommand{\arraystretch}{3}  % making height of table rows a little bigger for student responses
\begin{center}

\begin{tabular}{|c|m{4in}|}
\hline
\textbf{Characteristic} &  \\
\hline
domain & $\{ x \mid x \in \mathbb{R} \}$\\
\hline
range & $\{ y \mid y \in \mathbb{R}, y < 2 \}$\\
\hline
zero(s) & x = 4, 2\\
\hline
y-intercept & (0,-4)\\
\hline
interval(s) of increase & none\\
\hline
interval(s) of decrease & \shortstack[l]{$f(x)$ decreasing on $(-\infty, 3)$ \\$f(x)$ decreasing on $[3, \infty)$}\\
\hline
discontinuities & none\\
\hline
symmetry & even symmetry along vertical line $x=3$\\
\hline
end behaviours & \shortstack[l]{as $x \to \infty, f(x) \to -\infty$ \\ as $x \to -\infty, f(x) \to -\infty$}\\
\hline
\end{tabular}
\end{center}

A key characteristic of an absolute value function is that the domain is all
 real whole numbers. The range for the parent function of an absolute
value relationship is all non negative values but since f(x)
is reflected vertically and vertically translated 2 up, the range becomes all
real values less than and equal to 2. Zeroes are found by solving for x=0.
The y intercept is found by solving for x = 0 or simply identifying the vertical translation.
In the parent function of f(x) there are 2 positive intervals but since the function
is reflected vertically and translated 3 to the right, there are 2 decreasing intervals to the
left and right of x=3. The parent function of this relationship does not have any restrictions or characteristics
that cause discontinuities. Doing the even, odd and neither symmetry test with f(x), -f(x), f(-x) and -f(-x)
I determined that there is even symmetry and due to the horizontal translation mentioned before, the line
of symmetry is at x=3. Referencing the intervals both ends of the function are decreasing.

\newpage

\textbf{Check answer by graphing and comparing result to answers given:}
\vspace{1em}

From the graph you can see that the answers in the table above match the relatioship described by the graph below.
\vspace{2em}

\includegraphics[width=\linewidth]{A1 - 1 Proof.PNG}

%% I would recommend sandwiching your solution to every problem between the kind of structure I have provided below re: initial \vspace, the Solution: heading and the ending \vspace.
%\vspace{0.3cm} 
%\textbf{Solution:}\\
% Your solution starts here.
%\vspace{0.3cm}

\newpage

%% PROBLEM 2
\item \textbf{Solve} both inequalities.

\begin{enumerate}
\item Solve $|3x-5| \le 2$
\item Solve $-|-2x-1| < -4$
\end{enumerate}

\vspace{0.3cm} 
\textbf{Solution for (a):}
\vspace{-1em}

%solution
\addtolength{\jot}{1.2em}
\begin{align*}
    |3x-5| & \le 2 \\
    \bigg|3\left( x - \frac{5}{3} \right)\bigg| & \le 2 && \text{Factor by 3}\\
    |3| \bigg|x - \frac{5}{3} \bigg| & \le 2 && \text{1st property of absolute value}\\
    3\bigg| x - \frac{5}{3}\bigg| & \le 2 && \text{Since |3| = 3}\\
    \bigg| x - \frac{5}{3} \bigg| & \le \frac{2}{3} \numberthis &&  \text{divide by 3}\\
    -\frac{5}{3} - \frac{2}{3} & \leq x \leq -\frac{5}{3} + \frac{2}{3} && k - c \leq x \leq k + c \Longleftrightarrow |x-k| \leq c \\
    1 & \le x \le \frac{7}{3} && \text{simplify}
\end{align*}
$$ \therefore \text{The solutiont to the inequality} |3x-5| \le 2 \text{ is } \boxed{1 \le x \le \frac{7}{3}} $$

%proof/check
\vspace{0.5em}
\textbf{Check answer by starting from (1) to check the key points and use a line diagram to check the solution through logic:}
\vspace{0.2em}
\addtolength{\jot}{0em}
\begin{align*}
    \bigg|x - \frac{5}{3} \bigg| & = \frac{2}{3}\\
    x - \frac{5}{3} & = \pm \frac{2}{3} && |x| = c \longleftrightarrow x = \pm c\\
    x & = \frac{5}{3} \pm \frac{2}{3} && \text{add $\frac{5}{3}$ to both sides}
\end{align*}
\includegraphics[width=\linewidth]{A1-2 proof 1 (1).png}

\clearpage
\vspace{0.3cm} 
\textbf{Solution for (b):}
\vspace{-1.5em}

%solution
\addtolength{\jot}{0.3em}
\begin{align*}
    -|-2x-1| & < -4 && \text{\shortstack[l]{multiply all by -1, flip inequality sign \\ 3rd property of inequalities}} \\
    |-2x-1| & > 4 \\
    \bigg|-2 \left( x+\frac{1}{2} \right) \bigg| & > 4 && \text{factor -2} \\
    |-2|\bigg|x+\frac{1}{2}\bigg| & > 4 && \text{1st properoty of absolute values} \\
    2\bigg|x+\frac{1}{2}\bigg| & > 4 && |-2| = 2 \\
    \bigg|x+\frac{1}{2}\bigg| & > 2  \numberthis && \text{divide all by 2} \\
    x & < -\frac{1}{2} - 2 \text{ or } x > -\frac{1}{2} + 2 && |x-k| > c \Longleftrightarrow x < k-c \text{ or } x > k+c \\
    x & < -\frac{5}{2}, \text{\space} \frac{3}{2} < x && \text{simplify}
\end{align*}
$$ \therefore \text{The solutiont to the inequality} -|-2x-1| < -4 \text{ is } \boxed{-\frac{5}{2}, \text{\space} \frac{3}{2} < x} $$

%proof/check
\vspace{0.5em}
\textbf{Check answer by starting from (2) to check the key points and use a line diagram to check the solution through logic:}
\addtolength{\jot}{0em}
\begin{align*}
    \bigg|x+\frac{1}{2}\bigg| & = 2 \\
    x+\frac{1}{2} & = \pm 2 && |x| = c \longleftrightarrow x = \pm c \\
    x & = -\frac{1}{2} \pm 2 && \text{subtract } \frac{1}{2} \text{ }
\end{align*}
\includegraphics[width=\linewidth]{A1-2 proof 2 (1).png}

\newpage

%% PROBLEM 3
\item A 10 foot long stem of bamboo is broken in such a way that its tip touches the ground 3 feet away from the base of the stem. \textbf{Determine} the height of the break.


\begin{figure}[h]
\centering
\includegraphics[scale = 0.7]{bamboo.png}
\caption{Diagram from Problem 3}
\end{figure}

\vspace{0.2cm} 
\textbf{Question 3 Solution:}\\

Consider labeling all three sides of the triangle in the diagram given. The hypotenuse is c while the other sides are a and b. We know that a, or the bottom side, is 3 ft long.
\begin{center}
\includegraphics[scale = 0.2]{A1-3 Diagram.png}
\end{center}

From Pythagorean Theorum (PT) we know that $a^2 + b^2 = c^2$ or in this case $3^2 + b^2 = c^2$. 
We also know from the question that $b+c = 10$. Therefore, with 2 equations we can solve for the unknow values: b and c.

\vspace{-1.5em}
\begin{align*}
    \text{\space} c^2 & = \text{\space} b^2 + 9 && (1)\\[-1em]
    \text{\space} 10 & = \text{\space} b + c && (2)
\end{align*}

Rearrange the second equation $10 = b + c$ to isolate one of the variables to solve by substitution. I am isolating c to solve for b (the height).

\begin{align*}
    10 & = b + c \\
    \\[-4.5em]
    10 - b & = c && \text{(3) \space \space subtract b from all}
\end{align*}

\vspace{0.2em}
\begin{center}
    \textbf{Continue solution on next page}\\
\end{center}

\newpage

\vspace{0.2cm}
\begin{center}
    Now substitute (3) into (1) as c and solve for b:
\end{center}

\vspace{-2em}
\begin{align*}
    c^2 & = b^2 + 9 && \text{(1)} \\
    (10 - b)^2 & = b^2 + 9 && \text{(3) into (1) as c} \\
    100 -20b + b^2 & = b^2 + 9 && \text{expand square} \\
    100 - 20b & = 9 && \text{subtract} b^2 \text{from both sides} \\
    100 - 9 & = 20b && \text{subtract 9 and -20b from both sides} \\
    91 & = 20b && \text{simplify} \\
    b & = \frac{91}{20} && \text{divide 20 from both sides} \\ 
\end{align*}

\vspace{-4em}
\begin{center}
    $\therefore$ the height of the vertical bamboo shoot (break) is $\boxed{\frac{91}{20} \text{ft}}$ or 4.55 ft
\end{center}

%proof/check
\vspace{1em}
\begin{proof}
    \textbf{Check answer by solving for c with (2) and then confirming result with (1)}
    \begin{align*}
        10 - b & = c && \text{From (3)} \\
        c & = 10 - 4.55 && \text{substitute b for 4.55} \\
        c & = 5.45
    \end{align*}

    \begin{center}
        Now substitute b and c values into (1) to prove the equivalent values
    \end{center}

    \vspace{-1em}
    \begin{align*}
        c^2 & = b^2 + 9 && \text{From (1)}\\
        {(5.45)}^2 & = {(4.55)}^2 + 9 && \text{substitute the values of b and c} \\
        29.7025 & = 20.7025 + 9 && \text{expand and simplify} \\
        29.7025 & = 29.7025 \\
        \text{LHS} & = \text{RHS} && \qedhere
    \end{align*}
        
\end{proof}

\newpage

%% PROBLEM 4
\item \textbf{Define} (rewrite) $f(x) = |x^3 - x|$ as a piecewise function not including any expressions involving absolute value.

\vspace{0.3cm} 
\textbf{Solution for Question 4:}\\

%proof
Consider $f(x) = |x^3-x|$. The x intercepts (f(x) = 0) are found by converting the function to factor from.

\begin{align*}
    0 &= |x^3-x| \\
    &= |x(x^2-1)| && \text{factor x}\\
    &= |x(x+1)(x-1)| && (x^2-a^2) = (x+a)(x-a)\\
    &= |x||x+1||x-1| && |ab| = |a||b| \text{ 1st property of absolute value}\\
\end{align*}

\vspace{-2em}
From the calculations above, the x intercepts are at $x = 0, -1, 1$. Now Consider f(x) and it's non-absolute value equivalent, $g(x) = x^3-x$.
\vspace{-1em}

%graphic here
\begin{center}
    \includegraphics[scale = 0.5]{A1-4 diagram.png}
\end{center}

The 3 key points show us that there are 4 sections. The diagrams/Graphs shows that 2 of the 4 segments seperated by the x intercepts in g(x) represent negative values.
In f(x), the absolute value function, these segments are reflected vertically to ensure they are not negative values.
\vspace{1em}

\begin{center}
These segments are at: $ x \in (\infty, -1) $ and $ x \in (0, 1) $.

(I used $<$ instead of $\leq$ because 0 is not negative, therefore it does not need to be reflected)
\end{center}

$\therefore$ using the information gathered I wrote the function in piecewise form.
\vspace{1em}

%Piecewise Function Here
$$
f(x) =
\begin{cases}
    x^3-x & : 1 \leq x, \\
    & \text{\space} -1 \leq x \leq 0 \\
    -x^3+x & : x < -1 \\
    & \text{\space \space} 0 < x < 1
\end{cases}
$$

\vspace{1em}
We use $x^3-x$ to express  the segments that are positive and not reflected when comparing g(x) to f(x).
\vspace{2em}
Those parts of g(x) that are negative are positive in f(x) meaning there is a vertical reflection. 
This is represented by $-(x^3-x)$ where the ``-" at the front is the vertical reflection. 
In a simplified form, this is $-x^3+x$.

%proof/check
\vspace{3em}
\textbf{Check answer by graphing original f(x) and piecewise f(x) version:}
\vspace{-1.5em}
\begin{center}
    \includegraphics[scale=0.5]{A1-4 proof.png}
    Both graphs look identical and when placed on the same graph, they cover each other exactly.
\end{center}

\newpage

%% PROBLEM 5
\item \textbf{Determine} the inverse of the function $g(x)=\dfrac{-2}{x-1}+4$. \textbf{Prove} that $g$ and $g^{-1}$ satisfy the expression $g(g^{-1}(x))=x$.

%solution
\vspace{1em}
\textbf{Solution for Question 5: First we need to find the inverse of g(x) by swapping the dependant and independant variable}

\begin{align*}
    g(x) &=\dfrac{-2}{x-1}+4 \\
    y &=\dfrac{-2}{x-1}+4 && \text{replace g(x) with y}
\end{align*}

\begin{center}
    Now I swap x and y to create the inverse.
\end{center}

\begin{align*}
    x &=\dfrac{-2}{y-1}+4 && \text{x and y have been ``swapped"} \\
    x-4 &=\dfrac{-2}{y-1} && \text{subtracted 4 from both sides} \\
    (x-4)(y-1) &= -2 && \text{multiply al by y-1} \\
    xy-x-4y+4 &= -2 && \text{expand} \\
    xy-4y-x+4 &= -2 && \text{Rearrange}\\
    xy-4y &= -2-4+x && \text{subtract (-x+4) from both sides} \\
    y(x-4) &= -6+x && \text{factor y on the left side, simplify} \\
    y &= \frac{x-6}{x-4}
\end{align*}

$$\therefore \text{ the inverse of g(x) is: } \boxed{g^{-1}(x) = \frac{x-6}{x-4}} $$

\vspace{3em}
\begin{center}
    \textbf{Solution continues on next page}
\end{center}

\newpage
\begin{proof}
    \begin{center}
        \textbf{Now prove that $\boldsymbol{g(g^{-1}(x)) = x}$ with the given function of g(x):}
    \end{center}

    \begin{align*}
        g(x) &= \frac{-2}{x-1}+4\\
        g^{-1}(x) &= \frac{x-6}{x-4} \\
    \end{align*}
    \vspace{-5em}

    \begin{center}
        Given that $x = g(g^{-1}(x))$ and the know equations of g(x) and it's inverse:
    \end{center}
    \vspace{-2em}

    \begin{align*}
        x &= g(g^{-1}(x)) \\
        &= \frac{-2}{g^{-1}(x)-1}+4 && \text{substitute $g(x)$ as $\frac{-2}{x-1}+4$} \\
        &= \frac{-2}{\frac{x-6}{x-4}-1}+4 && \text{substitute $g^{-1}(x)$ as $\frac{x-6}{x-4}$} \\
        &= \frac{-2}{\frac{x-6}{x-4}-\frac{x-4}{x-4}}+4 && \text{1 rule: } \frac{x-4}{x-4} = 1 \\
        &= \frac{-2}{\frac{x-6-x+4}{x-4}}+4 && \text{simplify} \\
        &= \left( \frac{-2}{\frac{-2}{x-4}} \right) +4 \\
        &= \left( \frac{-2}{1} \times \frac{x-4}{-2} \right) +4 && \text{divide both sides by } \\
        &= x-4+4 && \text{simplify} \\
        x &= x \\
        LHS &= RHS
    \end{align*}

    \begin{center}
        $\therefore$ Since LHS = RHS, $g(g^{-1}(x)) = x$. \qedhere
    \end{center}
\end{proof}
\vspace{2em}
\begin{center}
    \textbf{proof/check on next page}
\end{center}
\newpage

\textbf{Check my work by using an example point:}
\vspace{1em}

I took an arbitrary x value, x=0, and found the coordinate of x=0 on $g(x)$.

\begin{align*}
    g(0) &= \frac{-2}{0-1}+4 \\
    &= 6
\end{align*}

The coordinate at x=0 is (0,6). We know that the inverse of a relationship switches the values on the domain and range.
This means that if (0,6) is a point on $g(x)$, (6,0) should be a point on $g^{-1}(x)$.

\begin{align*}
    g^{-1}(x) &= \frac{x-6}{x-4} \\
    g^{-1}(6) &= \frac{6-6}{6-4} \\
    &= \frac{0}{2} \\
    &= 0 \\
\end{align*}
\vspace{-3em}

Therefore, following the logic, the inverse of $g(x)$ found by this solution is in fact the inverse.
\newpage

\end{enumerate}
\end{document} 
