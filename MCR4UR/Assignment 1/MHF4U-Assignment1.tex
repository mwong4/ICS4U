%%
%%
\documentclass[12pt]{book}
\usepackage[utf8]{inputenc}
\usepackage{amsfonts,amssymb,amsmath,amsthm}
\usepackage{graphicx}
\usepackage{hyperref}
\usepackage{boxedminipage}
\usepackage{lastpage}
\usepackage{graphicx}
\usepackage{caption}
\usepackage{setspace}
\usepackage{polynom}
\usepackage{hyperref}
\usepackage{array}
\usepackage{geometry}
\newcolumntype{C}[1]{>{\centering\let\newline\\\arraybackslash\hspace{0pt}}m{#1}}
\setlength{\textheight}{10in}
\setlength{\textwidth}{7.4in}
\setlength{\topmargin}{-0.75in}
\setlength{\oddsidemargin}{-0.5in}
\setlength{\evensidemargin}{-0.5in}
\setlength{\parskip}{0.15in}
\setlength{\parindent}{0in}


\begin{document}

\vspace{-1.0in}\begin{center}
\Large{Advanced Functions }

\Large{Assignment \#1}


\end{center}

%\medskip

\vspace{0.015in}\hrulefill\ 

\textbf{Reference Declaration} %  Fill in your Reference Declarations in this section before your submit your assignment.

Complete the Reference Declaration section below in order for your assigment to be graded.

If you used any references beyond the course text and class notes (such as other texts, discussions with peers or online resources), indicate this information in the space below.  If you did not use any aids then explicitly state this in the space provided. 

Be sure to cite appropriate theorems throughout your work. You may use shorthand for well-known theorems like the FT (Factor Theorem), RRT (Rational Root Theorem), etc. 

Note: Your submitted work must be \textbf{your original work}. 

Family Name: Wong%Family Name Here
First Name: Max%First Name Here

Declared References: 

% Type your references here.
% You can use as many lines as required.

\vspace{0.015in}\hrulefill\ 

\newpage


% INSTRUCTIONS SECTION
\section*{Instructions}
\begin{center}
\setlength{\fboxrule}{2pt}
\begin{boxedminipage}{6.5in}
1.	Organize and express complete, effective and concise responses to each problem.\\
2.	Use appropriate mathematical conventions and notation wherever possible.\\
3.	Provide logical reasoning for your arguments and cite any relevant theorems. \\
4.  Ask your teacher questions if you need any clarification.
\end{boxedminipage}
\end{center} 

% EVALUATION SECTION
\section*{Evaluation}

% LEARNING EXPECTATION(S)
\begin{itemize}
\item[D3]	Students will compare the characteristics of functions, and solve problems by modelling and reasoning with functions.
\end{itemize}

% RUBRIC
\begin{tabular}{| C{2in} | C{1in} | C{1in} | C{1in} | C{1in} |}
\hline
\textbf{Criteria} & \textbf{Level 1} & \textbf{Level 2} & \textbf{Level 3} & \textbf{Level 4} \\
\hline
\emph{Understanding of Mathematical Concepts} & Demonstrates limited understanding & Demonstrates some understanding & Demonstrates considerable understanding & Demonstrates thorough understanding of concepts \\
\hline
\emph{Selecting Tools and Strategies} & Selects and applies appropriate tools and strategies, with major errors, omissions, or mis-sequencing & Selects and applies appropriate tools and strategies, with minor errors, omissions, or mis-sequencing & Selects and applies appropriate tools and strategies accurately, and in a logical sequence & Selects and applies appropriate and efficient tools and strategies accurately to create mathematically elegant solutions \\
\hline
\emph{Reasoning and Proving} & Inconsistently or erroneously employs logic to develop and defend statements & Statements are developed and defended with some omissions or leaps in logic & Frequently develops and defends statements with reasonable logical justification & Consistently develops and defends statements with sophisticated and/or complete logical justification \\
\hline
\emph{Communicating} & Expresses and organizes mathematical thinking with limited effectiveness & Expresses and organizes mathematical thinking with some effectiveness & Expresses and organizes mathematical thinking with considerable effectiveness & Expresses and organizes mathematical thinking with a high degree of effectiveness \\
\hline
\end{tabular}

\pagebreak



%%%%%%%%%%%% PROBLEMS START HERE

\begin{enumerate}

%% PROBLEM 1
\item  \textbf{Describe} the characteristics of the function $f(x) = -2|x-3|+2$ by filling in the table given below. \textbf{Write} a paragraph briefly explaining how you determined each characteristic.\\

\vspace{2.5in}

\renewcommand{\arraystretch}{3}  % making height of table rows a little bigger for student responses
\begin{center}

\begin{tabular}{|c|m{4in}|}
\hline
\textbf{Characteristic} &  \\
\hline
domain & \\
\hline
range & \\
\hline
zero(s) & \\
\hline
y-intercept & \\
\hline
interval(s) of increase & \\
\hline
interval(s) of decrease & \\
\hline
discontinuities & \\
\hline
symmetry & \\
\hline
end behaviours & \\
\hline
\end{tabular}
\end{center}


%% I would recommend sandwiching your solution to every problem between the kind of structure I have provided below re: initial \vspace, the Solution: heading and the ending \vspace.
%\vspace{0.3cm} 
%\textbf{Solution:}\\
% Your solution starts here.
%\vspace{0.3cm}

\newpage

%% PROBLEM 2
\item \textbf{Solve} both inequalities.

\begin{enumerate}
\item Solve $|3x-5| \le 2$
\item Solve $-|-2x-1| < -4$
\end{enumerate}

\newpage

%% PROBLEM 3
\item A 10 foot long stem of bamboo is broken in such a way that its tip touches the ground 3 feet away from the base of the stem. \textbf{Determine} the height of the break.


\begin{figure}[h]
\centering
\includegraphics{bamboo.png}
\caption{Diagram from Problem 3}
\end{figure} 

\newpage

%% PROBLEM 4
\item \textbf{Define} (rewrite) $f(x) = |x^3 - x|$ as a piecewise function not including any expressions involving absolute value.


\newpage

%% PROBLEM 5
\item \textbf{Determine} the inverse of the function $g(x)=\dfrac{-2}{x-1}+4$. \textbf{Prove} that $g$ and $g^{-1}$ satisfy the expression $g(g^{-1}(x))=x$.

\newpage

\end{enumerate}
\end{document} 
