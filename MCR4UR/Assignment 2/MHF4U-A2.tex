 %%
%%
\documentclass[12pt]{book}
\usepackage{amsfonts}
\usepackage{amsmath}
\usepackage{amssymb}
\usepackage{graphicx}
\usepackage{hyperref}
\usepackage{polynom}
\setlength{\textheight}{10in}
\setlength{\textwidth}{7.4in}
\setlength{\topmargin}{-0.75in}
\setlength{\oddsidemargin}{-0.5in}
\setlength{\evensidemargin}{-0.5in}
\setlength{\parskip}{0.15in}
\setlength{\parindent}{0in}

%commands
\newcommand\numberthis{\addtocounter{equation}{1}\tag{\theequation}}

\begin{document}


\vspace{-1.0in}\begin{center}
\Large{MHF4U : Advanced Functions }

\Large{Assignment \#2}


\end{center}

%\medskip

\vspace{0.015in}\hrulefill\ 

\textbf{Reference Declaration} %  Fill in your Reference Declarations in this section before your submit your assignment.

Complete the Reference Declaration section below in order for your assigment to be graded.

If you used any references beyond the course text and lectures (such as other texts, discussions with colleagues or online resources), indicate this information in the space below.  If you did not use any aids, state this in the space provided. 

Be sure to cite appropriate theorems throughout your work. You may use shorthand for well-known theorems like the FT (Factor Theorem), RRT (Rational Root Theorem), etc. 

Note: Your submitted work must be \textbf{your original work}. 

Family Name: Wong\\%Family Name Here
First Name: Max%First Name Here

Declared References: 

Used this forum answer to number the an element (Question 1): \\
\href{https://tex.stackexchange.com/questions/42726/align-but-show-one-equation-number-at-the-end}{stackexchange align* but show one equation number at the end} \\
Used YouTube tutorial to do long division with polynomials (Question 1): \\
\href{https://www.youtube.com/watch?v=9JfhV2Dimms} \\

% Type your references here.
% You can use as many lines as required.

\vspace{0.015in}\hrulefill\ 

\newpage

%%%%%%%%%%%% PROBLEMS START HERE

\begin{enumerate}

%% PROBLEM 1
\item Given the volume of a rectangular box is $V(x) = x^3 + 6x^2 + 11x + 6$, if the length of the box is $x+3$ cm, and its width is $x+2$ cm then \textbf{determine} the height of the box. Also, \textbf{determine} the domain and range of the function $V$.

%% I would recommend sandwiching your solution to every problem between the kind of structure I have provided below re: initial \vspace, the Solution: heading and the ending \vspace.
%\vspace{0.3cm} 
%\textbf{Solution:}\\
% Your solution starts here.
%\vspace{0.3cm}

\vspace{0.3cm} 
\textbf{Solution to Question 1:}\\

 Considering v(x) has a degree of 3. We are also given the length and width, $x+3$ and $x+2$ respectively. This means that:

 $$(x+3)(x+2)(x+a) = v(x)$$

 Where $a \in \mathbb{R}$. We know that there are 3 factors because of the degree. Since:

 \begin{align*}
    v(x) &= x^3+6x^2+11x+6 \\
    \\
    \therefore (x+3)(x+2)(x+a) &= x^3+6x^2+11x+6 && \text{Combining the previously listed equations} && \textbf{(1)}
 \end{align*}

 Dividing both sides by $(x+3)(x+2)$:

 \addtolength{\jot}{1em}
 \begin{align*}
    (x+3)(x+2)(x+a) &= x^3+6x^2+11x+6 \\
    (x+a) &= \frac{x^3+6x^2+11x+6}{(x+3)(x+2)} \\
    (x+a) &= \frac{x^3+6x^2+11x+6}{x^2+5x+6} && \text{Expanding denominator bracket} \\
 \end{align*}
 
 \vspace{-3.5em}
 \begin{center}
     Now, using long division, find the final missing value factor, $(x+a)$:
 \end{center}
 \vspace{-2em}

%Do long division
$$\polylongdiv{x^3+6x^2+11x+6}{x^2+5x+6}$$

\begin{center}
    $\boxed{\therefore \text{the height of the box is x+1}}$
    \vspace{1em}

    Also since this relationship is of degree 3, it is a cubic function. Cubic functions have the properties of a domain and range of all real whole numbers.
   
    $\boxed{\therefore D \in \mathbb{R}, R \in \mathbb{R}}$
\end{center}

\newpage

\textbf{Check work for question 1 by graphing and using a logical check:}
\vspace{1em}
From before, we stated that $(x+3)(x+2)(x+a) = x^3+6x^2+11x+6$. Now that we know the value of a (1), redo the operation to check the work.

%Do Math Here
\begin{align*}
    x^3+6x^2+11x+6 &= (x+3)(x+2)(x+a)&& \text{From (1)}\\
    x^3+6x^2+11x+6 &= (x+3)(x+2)(x+1) && a = 1\\
    &= (x^2+5x+6)(x+1) && \text{multiply first 2 factors and simplify} \\
    &= x^3+6x^2+11x+6 && \text{expand and simplify}\\
    LHS &= RHS && \therefore \text{height is correct}
\end{align*}

\vspace{1em}
From the graph, we can also see that domain and range are all real whole numbers.

%Graph here
\vspace{1.5em}
\includegraphics[width=\linewidth]{a2-1 graph.PNG}

\newpage

%% PROBLEM 2
\item Consider the quartic polynomial function $f(x) = x^4 - 5x^3 + x^2 + 21x - 18$. Given that there is a local minimum on $f(x)$ at $(-1,-32)$, use a logical argument to \textbf{prove} that this must the absolute minimum of $f(x)$ without graphing the function. \\

Recall that $x=c$, where $c \in \mathbb{R}$, corresponds to a \emph{local minimum} of a function $f(x)$ on an interval $I = (a,b)$ which contains $c$ provided that for every value of $x$ on the interval $I$ we have that $f(c) \le f(x)$.

\vspace{0.3cm} 
\textbf{Solution:}\\
 If we can create a factor table from this function we can better visualize 
 the relaitonship and better prove the local minimum. First factor f(x) by 
 finding the values of x where f(x) = 0. From experimentations:

 \vspace{-0.8cm}
 \begin{align*}
    f(1) &= 0 \\
    f(3) &= 0 \\
    f(-2) &= 0
 \end{align*}

To find the 4th missing factor, divide f(x) by the current 3 known factors found:

\begin{align*}
    \text{missing factor} &= \frac{x^4 - 5x^3 + x^2 + 21x - 18}{(x-1)(x-3)(x+2)} \\
    &= \frac{x^4 - 5x^3 + x^2 + 21x - 18}{(x^2-4x+3)(x+2)} && \text{expand and simplify denominator}\\
    &= \frac{x^4 - 5x^3 + x^2 + 21x - 18}{(x^3+2x^2-4x^2-8x+3x+6} \\
    &= \frac{x^4 - 5x^3 + x^2 + 21x - 18}{(x^3-2x^2-5x+6} && \text{simplified}
\end{align*}

\vspace{0.2cm}
\begin{center}
    Now use long division
\end{center}
\vspace{-0.5cm}
$$\polylongdiv{x^4 - 5x^3 + x^2 + 21x - 18}{x^3-2x^2-5x+6}$$

\newpage

$\therefore$ in factor form $f(x) = {(x-3)}^2(x-1)(x+2)$. Since f(x) is positive 
(no negative vertical reflection) and utilizing the x intercepts given by the factor form:

\begin{center}
    \begin{tabular}{|c|c|c|c|c|}
        \hline
        & ($-\infty$, -2) & (-2, 1) & (1, 3) & (3, $\infty$) \\ \hline
        x + 2 & $-$ & + & + & + \\ \hline
        x - 1 & - & - & + & + \\ \hline
        x - 3 & - & - & - & + \\ \hline
        x - 3 & - & - & - & + \\ \hline
        f(x) & + & - & + & + \\ \hline
    \end{tabular}
\end{center}

Using the intervals from the factor table above and the known propterties 
of linear and quadratic factors, we can create a rough diagram:

%diagram here
\begin{center}
    \includegraphics[scale=0.06]{A2-2 Diagram.jpeg}
\end{center}

\vspace{0.5cm}
We can agree that the minimum points are at the lowest points, and theoretically 
with a positive quadric function there are two possible negative areas where the minimum
is likely to reside within (since the negative interval sections are the lowest points). 
On the diagram these spots are the areas where rate of change goes from negative to 
positive (angled down proceeded by an angle up).

\vspace{0.5cm}
Lets check these two candidates. One of the possible minimums, at x=3, the line "skidms"
the x-axis. This means that this segment does not actually drop into the negative parts 
of the cartesian plane. This means that the only area that goes negative, at interval (-2, 1)
is where the local minimum must reside. Since the x value x=-1 of the coordinate (-1, -32) is inside
the previously identified interval, w can conclude that this coordinate is likely the local minimum.

\vspace{0.5cm}
\textbf{Check work for question 2 by using derivatives:}

\vspace{0.3cm}

\newpage

%% PROBLEM 3
\item Recall that a prime number is defined as an integer $n>1$ such that its only positive divisors are 1 and $n$. Let $\mathbb{P}$ represent the set of prime numbers. Suppose that you know that when you divide $g(x) = x^3 + 2x^2 + cx + d$ by $x-2$ you obtain a remainder of 14, \textbf{determine} the specific values of $c$ and $d$ given that $c \in \mathbb{Z}$ and $d \in \mathbb{P}$. 

\vspace{0.3cm} 
\textbf{Solution:}\\
 Your solution starts here.
\vspace{0.3cm}

\newpage

%% PROBLEM 4
\item Solve $\dfrac{7}{x+2} + \dfrac{5}{x-2} = \dfrac{10x-2}{x^2 - 4}$.

\vspace{0.3cm} 
\textbf{Solution:}\\
 Your solution starts here.
\vspace{0.3cm}

\newpage

%% PROBLEM 5
\item Solve $\bigg|\dfrac{x-4}{x+5}\bigg| \le 4$.

\vspace{0.3cm} 
\textbf{Solution:}\\
 Your solution starts here.
\vspace{0.3cm}

\newpage

\end{enumerate}
\end{document} 
