 %%
%%
\documentclass[12pt]{book}
\usepackage{amsfonts}
\usepackage{amsmath}
\usepackage{amssymb}
\usepackage{graphicx}
\usepackage{hyperref}
\usepackage{polynom}
\usepackage{amsthm}
\setlength{\textheight}{10in}
\setlength{\textwidth}{7.4in}
\setlength{\topmargin}{-0.75in}
\setlength{\oddsidemargin}{-0.5in}
\setlength{\evensidemargin}{-0.5in}
\setlength{\parskip}{0.15in}
\setlength{\parindent}{0in}

%commands
\newcommand\numberthis{\addtocounter{equation}{1}\tag{\theequation}}

\begin{document}


\vspace{-1.0in}\begin{center}
\Large{MHF4U : Advanced Functions }

\Large{Assignment \#2}


\end{center}

%\medskip

\vspace{0.015in}\hrulefill\ 

\textbf{Reference Declaration} %  Fill in your Reference Declarations in this section before your submit your assignment.

Complete the Reference Declaration section below in order for your assigment to be graded.

If you used any references beyond the course text and lectures (such as other texts, discussions with colleagues or online resources), indicate this information in the space below.  If you did not use any aids, state this in the space provided. 

Be sure to cite appropriate theorems throughout your work. You may use shorthand for well-known theorems like the FT (Factor Theorem), RRT (Rational Root Theorem), etc. 

Note: Your submitted work must be \textbf{your original work}. 

Family Name: Wong\\%Family Name Here
First Name: Max%First Name Here

Declared References: 

Used this forum answer to number the an element (Question 1): \\
\href{https://tex.stackexchange.com/questions/42726/align-but-show-one-equation-number-at-the-end}{https://tex.stackexchange.com/questions/42726/align-but-show-one-equation-number-at-the-end} \\
Used YouTube tutorial to do long division with polynomials (Question 1 and 2): \\
\href{https://www.youtube.com/watch?v=9JfhV2Dimms}{https://www.youtube.com/watch?v=9JfhV2Dimms} \\
Used Desmos for graphs

% Type your references here.
% You can use as many lines as required.

\vspace{0.015in}\hrulefill\ 

\newpage

%%%%%%%%%%%% PROBLEMS START HERE

\begin{enumerate}

%% PROBLEM 1
\item Given the volume of a rectangular box is $V(x) = x^3 + 6x^2 + 11x + 6$, if the length of the box is $x+3$ cm, and its width is $x+2$ cm then \textbf{determine} the height of the box. Also, \textbf{determine} the domain and range of the function $V$.

%% I would recommend sandwiching your solution to every problem between the kind of structure I have provided below re: initial \vspace, the Solution: heading and the ending \vspace.
%\vspace{0.3cm} 
%\textbf{Solution:}\\
% Your solution starts here.
%\vspace{0.3cm}

\vspace{0.3cm} 
\textbf{Solution to Question 1:}\\

 Considering v(x) has a degree of 3. We are also given the length and width, $x+3$ and $x+2$ respectively. This means that:

 $$(x+3)(x+2)(x+a) = v(x)$$

 Where $a \in \mathbb{R}$. We know that there are 3 factors because of the degree. Since:

 \begin{align*}
    v(x) &= x^3+6x^2+11x+6 \\
    \\
    \therefore (x+3)(x+2)(x+a) &= x^3+6x^2+11x+6 && \text{Combining the previously listed equations} && \textbf{(1)}
 \end{align*}

 Dividing both sides by $(x+3)(x+2)$:

 \addtolength{\jot}{1em}
 \begin{align*}
    (x+3)(x+2)(x+a) &= x^3+6x^2+11x+6 \\
    (x+a) &= \frac{x^3+6x^2+11x+6}{(x+3)(x+2)} \\
    (x+a) &= \frac{x^3+6x^2+11x+6}{x^2+5x+6} && \text{Expanding denominator bracket} \\
 \end{align*}
 
 \vspace{-3.5em}
 \begin{center}
     Now, using long division, find the final missing value factor, $(x+a)$:
 \end{center}
 \vspace{-2em}

%Do long division
$$\polylongdiv{x^3+6x^2+11x+6}{x^2+5x+6}$$

    $$\boxed{\therefore \text{the height of the box is x+1}}$$
    \vspace{1cm}
    \begin{center}
        Second part of solution on next page
    \end{center}
    \newpage

    Also since this relationship is of degree 3, it is a cubic function. Cubic functions have should 
    the properties of a domain and range of all real whole numbers but since the dimensions of a 
    of a 3D shape (Domain) cannot be negative in reality, the domain can only be 
    equal to or greater than zero. When the domain is at zero, or f(0), we can calculate the 
    minimum range value. 

    $$f(0) = 0^3+6\times 0^2+11\times 0+6 = 6$$
   
    $$\boxed{\therefore D: \{ x \in \mathbb{R} | x \geq 0 \} , R: \{ y \in \mathbb{R} | y \geq 6 \}}$$

\vspace{1cm}
\textbf{Check work for question 1 by graphing and using a logical check:}
\vspace{1em}
From before, we stated that $(x+3)(x+2)(x+a) = x^3+6x^2+11x+6$. Now that we know the value of a (1), redo the operation to check the work.

%Do Math Here
\begin{align*}
    x^3+6x^2+11x+6 &= (x+3)(x+2)(x+a)&& \text{From (1)}\\
    x^3+6x^2+11x+6 &= (x+3)(x+2)(x+1) && a = 1\\
    &= (x^2+5x+6)(x+1) && \text{multiply first 2 factors and simplify} \\
    &= x^3+6x^2+11x+6 && \text{expand and simplify}\\
    LHS &= RHS && \therefore \text{height is correct}
\end{align*}

\vspace{1em}
From the graph, we can also see that domain and ranges of the given function 
visually matches those determined by the solution.

%Graph here
\vspace{1.5em}
\includegraphics[width=\linewidth]{a2-1 graph.PNG}

\newpage

%% PROBLEM 2
\item Consider the quartic polynomial function $f(x) = x^4 - 5x^3 + x^2 + 21x - 18$. Given that there is a local minimum on $f(x)$ at $(-1,-32)$, use a logical argument to \textbf{prove} that this must be the absolute minimum of $f(x)$ without graphing the function. \\

Recall that $x=c$, where $c \in \mathbb{R}$, corresponds to a \emph{local minimum} of a function $f(x)$ on an interval $I = (a,b)$ which contains $c$ provided that for every value of $x$ on the interval $I$ we have that $f(c) \le f(x)$.

\vspace{0.3cm} 
\textbf{Solution for question 2:}\\
\begin{proof}
 If we can create a sign table from this function we can better visualize 
 the relaitonship and better prove the local minimum. First factor f(x) by 
 finding the values of x where f(x) = 0. We know that factors are typically between
 -6 and 6. From the Rational Root Theorum we can also narrow down our search by using
 factors of the last term divided by the factors of the first term, plus minus. Possible 
 candidates include x= -6, -3, -2, -1, 1, 2, 3 adnd 6.From experimentations:

 \vspace{-0.8cm}
 \begin{align*}
    f(1) &= 0 \\
    f(3) &= 0 \\
    f(-2) &= 0
 \end{align*}

To find the 4th missing factor, divide f(x) by the current 3 known factors found:

\begin{align*}
    \text{missing factor} &= \frac{x^4 - 5x^3 + x^2 + 21x - 18}{(x-1)(x-3)(x+2)} \\
    &= \frac{x^4 - 5x^3 + x^2 + 21x - 18}{(x^2-4x+3)(x+2)} && \text{expand and simplify denominator}\\
    &= \frac{x^4 - 5x^3 + x^2 + 21x - 18}{(x^3+2x^2-4x^2-8x+3x+6} \\
    &= \frac{x^4 - 5x^3 + x^2 + 21x - 18}{(x^3-2x^2-5x+6} && \text{simplified}
\end{align*}

\vspace{0.2cm}
\begin{center}
    Now use long division
\end{center}
\vspace{-0.5cm}
$$\polylongdiv{x^4 - 5x^3 + x^2 + 21x - 18}{x^3-2x^2-5x+6}$$

\newpage

$\therefore$ in factor form $f(x) = {(x-3)}^2(x-1)(x+2)$. Since f(x) is positive 
(no negative vertical reflection) and utilizing the x intercepts given by the factor form:

\begin{center}
    \begin{tabular}{|c|c|c|c|c|}
        \hline
        & ($-\infty$, -2) & (-2, 1) & (1, 3) & (3, $\infty$) \\ \hline
        x + 2 & $-$ & + & + & + \\ \hline
        x - 1 & - & - & + & + \\ \hline
        x - 3 & - & - & - & + \\ \hline
        x - 3 & - & - & - & + \\ \hline
        f(x) & + & - & + & + \\ \hline
    \end{tabular}
\end{center}

\vspace{0.5cm}
We can agree that the minimum points have the lowest values, and theoretically 
with a positive quadric function, the function must start in the top left and come down to form the shape "W".  
There are two possible negative areas where the minimum is likely to reside within 
(since the negative interval sections are the lowest points). 
With a positive polynomial of degree 3, the "W" shape has two troughs beside 
the triangle in the middle, which are the lowest points (the turning points, from down to up).

\vspace{0.5cm}
Lets check these two candidates. One of the possible minimums, where x=3, the line must "skims"
the x-axis. This means that this segment does not actually drop into the negative parts 
of the cartesian plane. This also means that the only area that goes negative, at interval (-2, 1),
is where the local minimum must reside. Since the x value x=-1 of the coordinate (-1, -32) is inside
the previously identified interval, we can conclude that this coordinate is likely the local minimum.

\vspace{0.8cm}
\begin{center}
    We can form a mental image similar to the following

    \includegraphics[scale=0.05]{A2-2 Diagram.jpeg}
\end{center}

\vspace{0.5cm}
There is a possibility that 2 seperate local minimums can reside within the negative interval but 
there must be a turn point (local maximum) that goes from up to down between the two local minimums. The properties do not 
support this idea because of the following logic:

\vspace{0.5cm}
Properties: (1) We know that the function degree is 3 meaning that there are 3 and only 3 turning points. 
(2) We also know that the relationship "skims" the x-axis only at the farthest right of the three intercepts:
at x=3, since only the x-3 factor is squared and $3>1>-2$. (3) From the sign table that at x=3, it is positive, 
therefore the "skim" point must be positive. (4) We also know from the facotred form that there are 3 intercepts.

\vspace{0.5cm}
Remember, the local maximum must be less than or equal to zero and respect the properties
to indicate a second local minimum.

\newpage

NOTE: All cases are expressing the function from left to right with "positive" indicating $y > 0$ and "negative" indicating $y < 0$. 
All examples start from the top left corner as a property of a positive polynomial function of degree 3.
\vspace{0.5cm}

Case 1: If the function goes down (from positive to negative) then back up and skims the x-axis (y=0) on 
the negative side, forming the first local minimum before dipping back down and up (all the way into the positives) 
to form a second local minimum and continuing positively forever. The "skims" happens in the negative instead of the positive, violating the third property.
The "skims" happens on the middle x-axis intercept in the middle at x=1 and not at the right most intercept, violating the second property.
\vspace{0.5cm}

Case 2: If the function goes down (from positive to negative) then back up but stays negative, not touching the x-axis before going 
back down before going back up (all the way into the positives), forming a second local minimum before going positively infinitly, there would be only 2 intercepts 
as opposed to the required 3 intercepts, violating the fourth property.
\vspace{0.5cm}

Case 3: If the function goes down (from positive to negative) then back up but stays negative, not touching the x-axis before going 
back down before going back up (all the way into the positives), forming a second local minimum, then going down and skimming the 
x-axis on the positive side proceeded by the direction upwards infinitly. This does satisfy (2) and (3) bur violates the third property 
because there are 5 turning points instead of three.
\vspace{0.5cm}

Case 4: If the function goes down (from positive to negative) then back up but stays negative, not touching the x-axis before going 
back down before going back up, "skimming" the x-axis on the negative side before going down negatively infinitly. This case has 4 
turning points instead of three, violating the first property
\vspace{0.5cm}

You could imagine the cases looking something like the following:

\includegraphics[width=\linewidth]{A2-2 Cases.jpeg}

\vspace{0.5cm}

From the cases, we know that the properties of the given function do now allow 2 local minimums to exist within the single 
negative interval. Therefore there is only 1 local minimum within the only negative interval and since our point resides within this 
interval:

$$\boxed{\text{it is reasonable to assume that the point given is the absolute local minimum.}} \qedhere$$

\end{proof}

\newpage

%% PROBLEM 3
\item Recall that a prime number is defined as an integer $n>1$ such that its only positive divisors are 1 and $n$. Let $\mathbb{P}$ represent the set of prime numbers. Suppose that you know that when you divide $g(x) = x^3 + 2x^2 + cx + d$ by $x-2$ you obtain a remainder of 14, \textbf{determine} the specific values of $c$ and $d$ given that $c \in \mathbb{Z}$ and $d \in \mathbb{P}$. 

\vspace{0.3cm} 
\textbf{Solution to question 3:}

\vspace{0.2 cm}
 Since we know that $g(x)$ divided by $x-2$ gives a remainder of 14,
 We can compare try the division and compare the resulting polynomial remainder
with the real value of 14. Since we know a factor is at $x=2$ (from $x-2$) leads 
to a remainder or y value of 14, we can state that $g(2) = 14$.

\begin{align*}
    g(x) &= x^3 + 2x^2 + cx + d \\
    g(2) &= 2^3 + 2\times 2^2 + c\times 2 + d && \text{Substitute x as 2}\\
    g(2) &= 8 + 8 + 2c + d && \text{Simplify}\\
    g(2) &= 16 + 2c + d \\
    14 &= 16 + 2c + d && g(2) = 14\\
    -2 &= 2c + d && \text{Subtract 16 from both sides}\\
\end{align*}

\vspace{-1cm}

Now that we have the equation $-2 = 2c + d$ and we know that $d \in \mathbb{P}$ and $c \in \mathbb{Z}$, consider the following.

In the equation $-2 = 2c + d$, c is multiplied by 2. This means 
that no matter the value of c, if it is an integer, the value 
of 2c must be even. 2c and d are added together to also create 
an even number: -2. When are two numbers are added together to 
an even number both numbers must be even or both must be odd. 
This means if 2c is even, d must also be even. The only prime 
number that exists is 2 since all other numbers that are even 
are divisible by 2. Therefore d must be 2. From this and the 
equation, we can determine the value of c.

\vspace{-0.6cm}

\begin{align*}
    -2 &= 2c + d \\
    -2 &= 2c + 2 && \text{Substitute d as 2} \\
    -2 - 2 &= 2c && \text{subtract 2 to both sides}\\
    -4 &= 2c && \text{Simplify} \\
    \frac{-4}{2} &= c && \text{divide both sides by 2} \\
    -2 &= c
\end{align*}

$$\boxed{\therefore \text{the value of c and d are -2 and 2 respectively.}}$$

\newpage

\textbf{Check Answer for Question 3 by substituting the know values of c and d back into the original equation as g(2) = 14:}

\begin{proof}
    \begin{align*}
        g(x) &= x^3 + 2x^2 + cx + d && \text{Original equation}\\
        g(x) &= x^3 + 2x^2 - 2x + 2 && \text{Since }c = -2, d = 2\\
        g(2) &= 2^3 + 2\times 2^2 - 2\times 2 + 2 && \text{Substitute x as 2}\\
        14 &= 2^3 + 2\times 2^2 - 2\times 2 + 2 && g(2) = 14\\
        14 &= 8 + 8 - 4 + 2\\
        14 &= 14 \\
        LHS &= RHS && \qedhere\\
    \end{align*}
\end{proof}

\vspace{-1cm}
$$\therefore \text{the values of c and d found are correct}$$

\newpage

%% PROBLEM 4
\item Solve $\dfrac{7}{x+2} + \dfrac{5}{x-2} = \dfrac{10x-2}{x^2 - 4}$.

\vspace{0.5cm} 
\textbf{Solution to question 4:}

\vspace{0.3cm} 
To solve isolate all values to one side of the equal sign before simplifying. Then, when fully simplified, take the numerator and solve for x.

\begin{align*}
    \dfrac{7}{x+2} + \dfrac{5}{x-2} &= \dfrac{10x-2}{x^2 - 4} \\
    \dfrac{7}{x+2} + \dfrac{5}{x-2} - \dfrac{10x-2}{x^2 - 4} &= 0 && \text{subtract } \dfrac{10x-2}{x^2 - 4} \text{ from both sides}\\
    \dfrac{7}{x+2} + \dfrac{5}{x-2} - \dfrac{10x-2}{(x+2)(x-2))} &= 0 && x^2-4 = (x+2)(x-2)\\
    \dfrac{7(x-2)}{(x+2)(x-2)} + \dfrac{5(x+2)}{(x-2)(x+2)} - \dfrac{10x-2}{(x+2)(x-2))} &= 0 && \text{Get common denominators} \\
    \dfrac{7(x-2)+5(x+2)-(10x-2)}{(x+2)(x-2)} &= 0 && \text{Add all fractions together}\\
    \dfrac{7x-14+5x+10-10x+2}{(x+2)(x-2)} &= 0 && \text{Expand and Simplify}\\
    \dfrac{2x-2}{(x+2)(x-2)} &= 0\\
\end{align*}

\vspace{-0.8cm} 
Now take the numerator and solve for x:
\vspace{-0.2cm} 

\begin{align*}
    \dfrac{2x-2}{(x+2)(x-2)} &= 0\\
    2x-2 &= 0\\
    2x &= 2 && \text{subtract 2 from both sides}\\
    \therefore x &= 1 && \text{divide all by 2}\\
\end{align*}

\vspace{-1cm} 
$$\boxed{\therefore \text{the solution to } \dfrac{7}{x+2} + \dfrac{5}{x-2} = \dfrac{10x-2}{x^2 - 4} \text{ is } x = 1}$$

\vspace{1cm} 
\begin{center}
    Proof is on next page
\end{center}

\newpage
\textbf{prove question 4 by substituting x=1 back into the original equation:}\\

%proof goes here
\begin{proof}
    \begin{align*}
        LHS &= \dfrac{7}{x+2} + \dfrac{5}{x-2} \\
        &= \dfrac{7}{1+2} + \dfrac{5}{1-2} && \text{Substitute x as 1} \\
        &= \dfrac{7}{3} + \dfrac{5}{-1} && \text{Simplify} \\
        &= \dfrac{7}{3} + \dfrac{-15}{3} \\
        LHS &= -\dfrac{8}{3}
    \end{align*}

    \begin{align*}
        RHS &= \dfrac{10x-2}{x^2 - 4} \\
        &= \dfrac{10\times 1-2}{1^2 - 4} && \text{Substitute x as 1} \\
        &= \dfrac{10-2}{1 - 4} && \text{Simplify}\\
        RHS &= -\dfrac{8}{3}\\
        RHS & =LHS && \qedhere
    \end{align*}
\end{proof}

$\therefore$ the answer is correct. 

\newpage

%% PROBLEM 5
\item Solve $\bigg|\dfrac{x-4}{x+5}\bigg| \le 4$.

\vspace{0.3cm} 
\textbf{Solution to question 5:}

\vspace{0.5cm}
 Consider the following identity of an absolute value inequality: $|x| \leq c \Longleftrightarrow -c \leq x \leq c$
 . Lets apply this to the given equation where c is 4 and x is the absolute value rational function.
 \vspace{-0.3cm}

 \begin{align*}
    \bigg|\dfrac{x-4}{x+5}\bigg| &\le 4 \\
    -4 \le \dfrac{x-4}{x+5} & \le 4 && \text{applying identity}
 \end{align*}

Now solve the inequality found as if it were composed of two seperate inequalities with the rational function being repeated twice. Split the inequality in the following fashion written below:

\begin{align*}
    -4 &\le \dfrac{x-4}{x+5} && (1)\\
    \dfrac{x-4}{x+5} &\le 4 && (2)
\end{align*}

\vspace{1cm}

%Solving (1)
Take (1), $-4 \le \dfrac{x-4}{x+5}$ and solve the inequality.

\begin{align*}
    -4 &\le \dfrac{x-4}{x+5} && (1)\\
    0 &\le \dfrac{x-4}{x+5} + 4 && \text{add 4 to both sides} \\
    0 &\le \dfrac{x-4}{x+5} + \dfrac{(4)(x+5)}{x+5} && \text{Get same denominator} \\
    0 &\le \dfrac{x-4}{x+5} + \dfrac{(4x+20}{x+5} && \text{Expand} \\
    0 &\le \dfrac{x-4+4x+20}{x+5} && \text{Add fractions} \\
    0 &\le \dfrac{5x+16}{x+5} && \text{Simplify} \\
\end{align*}

\newpage

\begin{center}
To solve the rational inequality, we create a sign table
 to determine the appropriate intervals.
\end{center}
\vspace{0.5cm}

\begin{center}
    \begin{tabular}{|r|c|c|c|}
        \hline
        & $(-\infty, -5)$ & $\left(-5, -\dfrac{16}{5} \right)$ & $\left( -\dfrac{16}{5}, \infty \right)$ \\ \hline
        5x+16 & - & - & + \\ \hline
        x+5 & - & + & + \\ \hline
        f(x) & + & - & + \\ \hline
    \end{tabular}
\end{center}

\vspace{0.3cm}

Since we are searching for values greater than or equal to zero,
 look for the intervals where the value is positiive or zero. We can see
  from the sign table that, when including the values at zero:

$$x \in (-\infty, -5], \left[ -\frac{16}{5}, \infty \right) \text{\space\space\space\space\space} (3)$$

\vspace{1cm}

%Solving (2)
Take (2), $\dfrac{x-4}{x+5} \le 4$ and solve the inequality.

\begin{align*}
    \dfrac{x-4}{x+5} &\le 4 && (2)\\
    \dfrac{x-4}{x+5} - 4 &\le 0 && \text{subtract 4 to both sides} \\
    \dfrac{x-4}{x+5} - \dfrac{(4)(x+5)}{x+5} &\le 0 && \text{Get same denominator} \\
    \dfrac{x-4}{x+5} - \dfrac{4x+20}{x+5} &\le 0 && \text{Expand} \\
    \dfrac{x-4-(4x+20)}{x+5} &\le 0 && \text{Combibe fractions} \\
    \dfrac{x-4-4x-20}{x+5} &\le 0 && \text{expand} \\
    \dfrac{-3x-24}{x+5} &\le 0 && \text{Simplify} \\
    \dfrac{3x+24}{x+5} &\geq 0 && \text{Multiply by -1 to all, flip inequality sign} \\
\end{align*}

\newpage

\begin{center}
To solve the rational inequality, we create a sign table
 to determine the appropriate intervals.
\end{center}
\vspace{0.5cm}

\begin{center}
    \begin{tabular}{|r|c|c|c|}
        \hline
        & $(-\infty, -8)$ & $(-8, -5)$ & $(-5, \infty)$ \\ \hline
        3x+24 & - & + & + \\ \hline
        x+5 & - & - & + \\ \hline
        f(x) & + & - & + \\ \hline
    \end{tabular}
\end{center}

\vspace{0.3cm}

Since we are searching for values greater than or equal to zero,
 look for the intervals where the value is positiive or zero. We can see
  from the sign table that, when including the values at zero:

$$x \in (-\infty, -8], [-5, \infty) \text{\space\space\space\space\space} (4)$$
\vspace{0.1cm}

Now let's combine the intervals found from (1) and (2). We can do this by 
plotting the intervals on a line diagram and look for the intervals where
(3) and (4) overlap. In essence we need to find intervals that satisfy both 
inequalities, which is why we look for overlap.

\includegraphics[width=\linewidth]{A2-5 Line Diagram (1).png}

From the line diagram, we can see that the 4 intervals overlap at 2 areas,
 highlighted by the green.
 
$$\boxed{\therefore \text{the intervals that satisfy the inequality} \bigg|\dfrac{x-4}{x+5}\bigg| \le 4 \text{ are } (-\infty , -8] \cup \left[ -\dfrac{16}{5}, \infty \right) \text{.}}$$

\vspace{2cm}
\begin{center}
    Proof is on the next page
\end{center}

\newpage

\textbf{Prove the intervals found in question 5 with a graph:}

\begin{center}
\includegraphics[scale = 0.7]{A2-5 Proof.PNG}
\end{center}

\vspace{0.5cm}
From the graph we can see that the intervals found seem to intervals
 on the graph that exist below or equal to y=4.

\newpage

\end{enumerate}

\end{document} 
