\documentclass{article}

\setlength{\parindent}{4em}
\setlength{\parskip}{1em}

\newcommand{\rt}{$\to\ $}

\begin{document}

\title{Pre-Read Notes}
\author{Max C Wong}
\maketitle

\tableofcontents
\clearpage

    %Chapter 1
    \section{Chapter 1: Function Characts. and Props.}

    \subsection{Functions}
    \begin{itemize}
        \item A relationship is a function if a all values on the domain have less than or equal to 1 value on the range
        \item circular motion is represented by sinosoidal functions
        \item functions can be represented in many ways
    \end{itemize}

    \subsection{Absolute value}
    \begin{itemize}
        \item $f(x) = |x|$ describes values $\geq 0$ 
    \end{itemize}

    \subsection{Properties}
    \begin{itemize}
        \item Each function has a unique mixture of elements, usually most visually apparent on a graph
        \item This can be used to distinguish them
    \end{itemize}

    \subsection{Sketching Graphs}
    \begin{itemize}
        \item Do transformations in steps
        \item Do translations last when listing transformations
        \item general formula: $y = af(k(x - d)) + c$
    \end{itemize}

    \subsection{Inverse}
    \begin{itemize}
        \item Inverse is done by swapping x and y variables
        \item graphically a reflection about x and y axis (along $y=x$)
        \item denoted by $f^{-1}(x)$
        \item not all inverses are functions
    \end{itemize}

    \subsection{Piecewise}
    \begin{itemize}
        \item A function with multiple rules
        \item Related to specific intervals in the domain
        \item filled circle for inclusive, empty circle for exclusive
        \item Does not have to be continuous 
    \end{itemize}

    \subsection{Operations within}
    \begin{itemize}
        \item If functions have overlapping domains they can be combined
        \item By combining the dependant variable in some way
        \item Properties carry onwards
    \end{itemize}

    %Chapter 3
    \section{Chapter 3: Polynomial Functions}

    \subsection{Polynomial Functions}
    \begin{itemize}
        \item A polynomial arranged in this formula
        \item $a_nx^n + a_{n-1}x^{n-1} + \cdots + a_2x^2 + a_1x + a_0$
        \item where n are whole numbers and a are real numbers
        \item most simplified form
        \item the ``degree" is the highest exponent in the polynomial
        \item degree is proportional to the number of ``lines/curves" in the graph
    \end{itemize}

    \subsection{Properties}
    \begin{itemize}
        \item P. function's degree can indicate a lot:
        \item shape, turning points, zeroes, and end behavior
        \item odd degree \rt opposite end dir., even degree \rt same end dir.
        \item if even
        \item if leading coefficient is pos \rt goes positive to negative
        \item if leading coefficient is neg \rt goes negative to positive
        \item if odd
        \item neg \rt face negative, pos \rt face positive
        \item turning points proportional to n - 1
        \item y axis symmetrical \rt even function, rotational symetry \rt odd function
    \end{itemize}

    \subsection{Factored Form}
    \begin{itemize}
        \item Polynomial function family \rt P. functions of similar properties
        \item zeroes of a P. function are same as roots of related P equation (when factored?)
        \item Factored form gives roots, factored form at 0 gives zeroes
        \item Use zeroes and a point to get equation from $f(x) = a(x - b)(x - b) \cdots$ where a is solved using the extra point and b, c, $\cdots$ are zeroes
        \item if root is exponent 1 \rt passes through as if linear
        \item if root is exponent 2 \rt glances off like quad vertex
        \item if root is exponent 3 \rt passes flat before going through, like parent root function
    \end{itemize}

    \subsection{Transformations}
    \begin{itemize}
        \item Like any other function
    \end{itemize}

    \subsection{Dividing}
    \begin{itemize}
        \item Polynomials can be divided in similar manner to numbers
        \item Like with long division
        \item remainders are added to the end of the equation, rest becomes factors
    \end{itemize}

    \subsection{Factoring}
    \begin{itemize}
        \item Remainder theorum: $\frac{f(x)}{x-1} = f(a)$
        \item Factor theorum: x - a is a factor if $f(a) = 0$
        \item To factor:
        \begin{enumerate}
            \item use factor theorum to determine factor
            \item divide by factor
        \end{enumerate}
    \end{itemize}

    \subsection{Factoring Sum or Difference}
    \begin{itemize}
        \item Expressions with two perfect cubes
        \item $A^3 + B^3 = (A+B)(A^2-AB+B^2)$
        \item $A^3 - B^3 = (A-B)(A^2+AB+B^2)$
    \end{itemize}

    %Chapter 4
    \section{Chapter 4: Polynomial Equ. and Ineq.}

    \subsection{Solving}
    \begin{itemize}
        \item Solutios of f(x) = 0 are zeroes
        \item sometimes you need to ignore the values outside of the defined intervals
    \end{itemize}

    \subsection{Solving Linear Inequalities}
    \begin{itemize}
        \item Solve linear inequalities by rearranging, like solving linear equations
        \item If you multiply or divide by a negative number, flip over the inequality sign 
    \end{itemize}

    \subsection{Solving Polynomial Inequalities}
    \begin{itemize}
        \item To solve:
        \begin{enumerate}
            \item Solve for main points, like roots
            \item Plot on some sort of line system
            \item This will give you your solution ranges
        \end{enumerate}
    \end{itemize}

    \subsection{Rates of Change in Polynomials}
    \begin{itemize}
        \item Rate of change is $\frac{change \:in \: range}{change \: in \: domain}$
        \item On interval $x_1 \leq x \leq x_2 $ is $\frac{f(x_2) - f(x_1)}{x_2 - x_1}$
        \item When x is very small, $roc = \frac{f(x + h) - f(x)}{h}$
        \item On any ``indexes" the roc is near 0
    \end{itemize}

    \section{Chapter 5}
    \section{Chapter 6}
    \section{Chapter 7}
    \section{Chapter 8}
    \section{Chapter 9}
    \section{Chapter 2}
\end{document}