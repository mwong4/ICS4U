\documentclass[12pt]{article}

\setlength{\parindent}{4em}
\setlength{\parskip}{1em}
\usepackage{amsfonts}

\begin{document}

$\times  \div  \equiv  \neq  \ne  \geq  \ge  \leq  
\le  \not\le  \forall  \mid  \exists  \in  \not\in  \notin  \ni  \pi  
\theta  \alpha  \beta  \rightarrow  \Rightarrow$

%lines
\section{lines}

$\mid | \setminus \backslash$
\\
$$ A \backslash B = \{ a \in A \mid a \not\in B \} $$
$$ |x|^2 = x^2, |x| \geq 0 $$

%1cm
\section{1cm}

$$ \textrm{1cm(} x,y \textrm{) = the smallest positivte integer } z \textrm{ so that } x \mid z \textrm{ and } y \mid z$$

%dyadic
\section{dyadic}

$$the dyadic rationals are \left\{ \frac{a}{2^b} \middle|\: a,b integral \right\}\footnote{Hello World}$$ 

%verbatim
\section{verbatim}

If you type in \verb+ \verb| leading spaces|+ you get \verb|leading spaces.|
\begin{verbatim}
With verbatim you can get blank lines:
like ^that^ one.
\end{verbatim}

\clearpage

%mathfonts
\section{mathfonts}

$\mathfrak{hi}$
$\mathbb{hi}$
$\mathcal{hi}$

The permutation group $\mathfrak{G_n}$ is defined as 
$\{ \pi \in \mathbb{Z}^n \mid 1 \leq \pi \leq n$, all $\pi_i$
 distinct $\}$ and has cardinality $n!$, while the power set
 $\mathcal{P}(n)$ is defined as the family of all subsets of $S$, 
 and has cardinailty $2^{|S|}$.

%bodyparts

\appendix

\section{appendix}



\end{document}