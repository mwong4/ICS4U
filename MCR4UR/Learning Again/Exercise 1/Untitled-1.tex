\documentclass{article}

\setlength{\parindent}{4em}
\setlength{\parskip}{1em}

\begin{document}

\title{Section 1}
\author{Max C Wong}
\maketitle

\tableofcontents

\newpage

\section{1-1}

%milk
\subsection{milk}

Would you like 2\% milk? Yes please :\}

The output \% is produced by the input \textbackslash\%, but \textbackslash is not produced by \textbackslash\textbackslash

\section{1-2}

%exponents
\subsection{exponents}

By convention, a tower of exponents $a^{b^c}$ in mathematics is always interpreted as $a^{(b^c)}$.
In other words, exponents are riught-associative. The reason for this is that the
left-associated interpretation ${(a^b)}^c$ can be expressed more simply as $a^{b\times c}$
, by the laws of exponents.

%limit
\subsection{limit}

$$\int^1_0 f(x)dx = \lim_{t \to \infty} t^{-1} \sum^t_{i=1}f(i/t)$$

%fractions
\subsection{fractions}

$$\frac{1}{\sqrt[4]{5} - \sqrt[4]{2}} = \frac{(\sqrt[4]{5} + \sqrt[4]{2})(\sqrt{5} + \sqrt{2})}{3}$$
$$1 + \frac{1}{1 + \frac{1}{1 + \frac{1}{1 + \cdots}}} = \frac{1 + \sqrt{5}}{2}$$

\section{1-3}

%lists
\subsection{lists}

\begin{enumerate}
    \item for the money
    \item for the show
    \item to get ready
    \begin{itemize}
        \item now
        \item go
        \item[$\diamond$] chat 
        \item go!
    \end{itemize}
\end{enumerate}

%alligator
\subsection{alligator}

Grandpa said, ``When I was your age --- in the `roaring '20s --- we had it rough. The `alligator 
shoes' we wore during the 1927--1928 school year were just to kick the gators tryin' to eat us."
 My dad chimed in, ``I wouldn't have minded your -40 tempatures instead of our +40 ones!"

%pate
\subsection{p\^ate}

\emph{En fran\c{c}ais,} \textbf{p\^ate} can mean either pastry or paste, its plural \textbf{p\^ates}
most commonly refers to pasta noodles, and \textbf{p\^at\'e} is a congealed meat product.


\end{document}

%Fixing errors (1-1)
%\documentclass{article}
%\begin{document}
%In \LaTeX, \textbackslash is the escape character.
%\end{document}