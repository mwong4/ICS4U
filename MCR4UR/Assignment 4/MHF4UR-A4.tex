 %%
%%
\documentclass[12pt]{book}
\usepackage{amsfonts}
\usepackage{amsmath}
\usepackage{amsthm}
\usepackage{amssymb}
\usepackage{graphicx}
\usepackage{hyperref}
\DeclareMathOperator\cis{cis} %defining cis as an function
\setlength{\textheight}{10in}
\setlength{\textwidth}{7.4in}
\setlength{\topmargin}{-0.75in}
\setlength{\oddsidemargin}{-0.5in}
\setlength{\evensidemargin}{-0.5in}
\setlength{\parskip}{0.15in}
\setlength{\parindent}{0in}

\begin{document}


\vspace{-1.0in}\begin{center}
\Large{MHF4U :  Advanced Functions }

\Large{Assignment \#4}


\end{center}

%\medskip

\vspace{0.015in}\hrulefill\ 

\textbf{Reference Declaration} %  Fill in your Reference Declarations in this section before your submit your assignment.

Complete the Reference Declaration section below in order for your assigment to be graded.

If you used any references beyond the course text and lectures (such as other texts, discussions with peers or online resources), indicate this information in the space below.  If you did not use any aids, state this in the space provided. 

Be sure to cite appropriate theorems throughout your work. You may use shorthand for well-known theorems like the FT (Factor Theorem), RRT (Rational Root Theorem), etc. 

Note: Your submitted work must be \textbf{your original work}. 

Family Name: Wong\\%Family Name Here
First Name: Max%First Name Here

Declared References: 

% Type your references here.
% You can use as many lines as required.

\vspace{0.015in}\hrulefill\ 

\vspace{0.5cm}
\textbf{[NOTE]: Law of Logarithms is represented with LOL}

\newpage

%%%%%%%%%%%% PROBLEMS START HERE

\begin{enumerate}

%% PROBLEM 1
\item \textbf{Determine} all real numbers $x$ that satisfy the equation $\log_3(x-7) + \log_3(x-9) = 1$.

\vspace{0.5cm}
\textbf{Solve question 1 by isolating equation to one sidem performing algebra before then factoring}

\vspace{0.2cm}
First, we should identify restrictions. We know that if the parent 
logarithm is $\log_a(b) = c$, b must be positive. Taking the two "b" values 
from the equation:

\vspace{-0.2cm}
\begin{align*}
    x-7 &> 0 && \text{Subtract 7 from both sides}\\
    x &> 7 \\
\end{align*}

\vspace{-1.5cm}
\begin{align*}
    x-9 &> 0 && \text{Subtract 9 from both sides}\\
    x &> 9 \\
\end{align*}
\vspace{-0.7cm}

Since $x > 9 > 7$, we can say that the restriction is: ${x| x \in \mathbb{R}, x > 9}$
Now we can start solving for x. Start by breaking the equation down into a form 
that can be factored before factoring.

\addtolength{\jot}{1em}
\begin{align*}
    \log_3(x-7) + \log_3(x-9) &= 1 \\
    \log_3((x-7)(x-9)) &= 1 && \text{LOL Product } log_a(xy) = \log_a(x) + \log_a(y)\\
    3^1 &= (x-7)(x-9) && \log_a(b) = c \Longleftrightarrow a^c = b\\
    3 &= (x-7)(x-9) && \text{Simplify}\\
    3 &= x^2-16x+63 && \text{Multiply factors, expand}\\
    0 &= x^2-16x+63-3 && \text{Subtract 3 from both sides}\\
    0 &= x^2-16x+60 && \text{Simplify}\\
    0 &= (x-6)(x-10) && \text{Factor}\\
    x &= 6, 10 \\
\end{align*}

Now we know that $x=6, 10$ but from the restriction, x must be greater 
than 9. Therefore the solution $x=6$ is inadmissable.

$$\boxed{\therefore \text{The solution to } \log_3(x-7) + \log_3(x-9) = 1 \text{ is } x=10}$$

\newpage

%% PROBLEM 2
\item After Judgement Day, crows rose to become the dominant species and took over the planet Earth. Initially, there were 40000 crows and with no competition for food sources, this population of crows tripled in population every 1.5 years. A "mega-murder" is a murder of crows with a population of one million.

\begin{enumerate}
\item \textbf{Model} the relationship between the population of crows and time since Judgement Day.
\item \textbf{Determine} after how many years is Earth ruled by a mega-murder.
\end{enumerate}

\vspace{0.5cm}
\textbf{Solve question 2(a) by first identifying the key characteristics for a exponential model}

\vspace{0.2cm}
Consider the following the general formula for exponential functions, where c(t) is the number 
of crows and t is the time in days since Judgement day:

$$c(t) = \text{initial}(1+\text{growth rate})^{\frac{t}{\text{period}}}$$

Let us determine the values required to definte the relationship from the 
data provided in the question.

\begin{center}
    \begin{tabular}{ll}
        Initial Population: & 40000 crows \\
        Rate of growth: & 200 \% increase per period \\
        & = +2 \\
        Period of growth & 1.5 years \\
    \end{tabular}
\end{center}

Now let's put this all together into the general formula above and simplify:

\begin{center}
    \begin{align*}
        c(t) &= \text{initial}(1+\text{growth rate})^{\frac{t}{\text{period}}} \\
        c(t) &= 40000(1+2)^{\frac{t}{1.5}} && \text{Substitute values in} \\
        c(t) &= 40000(3)^{\frac{t}{1.5}} && \text{Simplify}\\
    \end{align*}
\end{center}

\begin{center}
    $\therefore$ The model that describes the relationship between number of crows (c(t)) and days after Judgement day (t) is $\boxed{40000(3)^{\frac{t}{1.5}}}$
\end{center}

\vspace{1cm}

\begin{center}
    \textbf{Part b is on next page}
\end{center}

\newpage

\vspace{0.5cm}
\textbf{Solve question 2(b) by solving for the value of t}

\vspace{0.2cm}
We know that c(t) represents the number of magnificient crows and t represents the 
time in days since Judgement day. If we know that the number of crows or c(t) is a murder 
orone million, we can solve for t at c(t) = 1000000:

\begin{align*}
    c(t) &= 40000(3)^{\dfrac{t}{1.5}} && \text{From (a)} \\
    1000000 &= 40000(3)^{\dfrac{t}{1.5}} && \text{Sub 1000000 as c(t)} \\
    \frac{1000000}{40000} &= (3)^{\dfrac{t}{1.5}} && \text{Divide both sides by 40000} \\
    25 &= (3)^{\dfrac{t}{1.5}} && \text{Simplify} \\
    \log(25) &= \log((3)^{\dfrac{t}{1.5}}) && \text{Apply }\log_{10} \text{ to both sides} \\
    \log(25) &= \dfrac{t}{1.5}\log(3) && \text{LOL Power rule, } \log_a(r^b) = b\log_ar \\
    \log(5^2) &= \dfrac{t}{1.5}\log(3) && 25 = 5^2 \\
    2\log(5) &= \dfrac{t}{1.5}\log(3) && \text{LOL Power rule} \\
    \dfrac{2\log(5)}{log(3)} &= \frac{t}{1.5} && \text{Divide both sides by } log(3) \\
    \dfrac{2\log(5)}{log(3)} \times 1.5 &= t && \text{Multiply both sides by 1.5} \\
    \dfrac{3\log(5)}{log(3)} &= t && \text{Simplify, multiply 2 with 1.5} \\
    \dfrac{3\log(5)}{log(3)} &\approx 4.394  \\
\end{align*}

\begin{center}
    $\therefore$ it takes $\boxed{\dfrac{3\log(5)}{log(3)}}$ or approximately $\boxed{4.39}$ days since Judgement day 
    to reach a mega-murder (one million) of crows.
\end{center}

\newpage

%% PROBLEM 3
\item \textbf{Prove} that if $a,b,c$ are the first three terms of a geometric sequence then $\log(a), \log(b), \log(c)$ are the first three terms of an arithmetic sequence.

\vspace{0.5cm}
\textbf{Solution for question 3}

\vspace{0.2cm}
We know that the terms a, b and c are the first three terms of a geometric 
sequence. In a geometric sequence, there is a common rate of change or common 
multiplier between each term. If we express the common multiple as x, we know that
we can get a term n by multiplying term n-1 by x, where n is any positive integer:

\begin{align*}
    a \times x &= b && \text{Multiply first term by x to get second term}\\
    x &= \frac{b}{a} && (1)\text{ Divide both sides by a} \\
    b \times x &= c && \text{Multiply second term by x to get third term}
    x &= \frac{c}{b} && (2)\text{ Divide both sides by b} \\
\end{align*}

Since both (1) and (2) are equal to the same value, x, we can 
substitute on into the other as x and combine the two equation. 
I label this (3):

$$x = \frac{b}{a}, x = \frac{c}{b}$$
$$\frac{b}{a} = \frac{c}{b} \text{\space \space (3)}$$

Let's apply a similar logic to the logs. If we assune that log(a), log(b) and log(c) 
form a arithmetic sequence, then a common difference, y, can be added 
to every term of the sequence in order to determine the proceeding term. From this, 
we can form an equation and prove that this really is an arithmetic sequence.

\begin{align*}
    \log(a) + y &= \log(b) && \text{Add y to first term to get the second term} \\
    y &= \log(b) - \log(a) && (4)\text{ Subtract log(a) from both sides} \\
    \log(b) + y &= \log(c) && \text{Add y to second term to get the third term} \\
    y &= \log(c) - \log(b) && (5)\text{ Subtract log(b) from both sides} \\
\end{align*}

Since both (4) and (5) are equal to the same value, y, we can 
substitute on into the other as y and combine the two equation. 
I label this (6):

$$y = \log(b) - \log(a), y = \log(c) - \log(b)$$
$$\log(b) - \log(a) = \log(c) - \log(b) \text{\space \space (6)}$$

\newpage

Now, with (3) and (6), since we know (3) is true, if we can express (6) to (3), 
we can express show that (6) is also true, therefore proving the log(a), log(b) and log(c) 
form a arithmetic sequence.

\begin{align*}
    \frac{b}{a} &= \frac{c}{b} && (3) \\
    \log\left( \frac{b}{a} \right) &= \log\left( \frac{c}{b} \right) && \text{Apply } \log_10 \text{ to both sides} \\
    \log(b) - \log(a) &= \log(c) - \log(b) && \text{LOL quotient rule, } \log_r\left( \frac{m}{n} \right) = \log_r(m) - \log_r(n) \\
    & \equiv (6)
\end{align*}

From this, (3) = (6). Therefore, since (3) is true, (6) must also be true.

$$\boxed{\therefore \text{log(a), log(b), and log(c) form a arithmetic sequence.}}$$

\newpage


%% PROBLEM 4
\item Given $a,b,c \in \mathbb{R^+}$ with $a\neq1$, $b\neq1$, $c\neq1$ \textbf{prove} $$\dfrac{1}{1+\log_a(bc)} + \dfrac{1}{1+\log_b(ac)} + \dfrac{1}{1+\log_c(ab)} = 1$$.

\vspace{0.5cm}
\textbf{Solve Question 4 by showing that both sides of the equation are equivalent statements:}

\vspace{0.2cm}
Take the left hand side and manipulate it to create the right hand side:

\begin{proof}
    \begin{align*}
        LHS &= \dfrac{1}{1+\log_a(bc)} + \dfrac{1}{1+\log_b(ac)} + \dfrac{1}{1+\log_c(ab)} && \text{LHS = Left Hand Side} \\
        &= \dfrac{1}{1+\dfrac{\log(bc)}{\log(a)}} + \dfrac{1}{1+\dfrac{\log(ac)}{\log(b)}} + \dfrac{1}{1+\dfrac{\log(ab)}{\log(c)}} && \text{Change of base identity}\\
        &= \dfrac{1}{\dfrac{\log(a)}{\log(a)}+\dfrac{\log(bc)}{\log(a)}} + \dfrac{1}{\dfrac{\log(b)}{\log(b)}+\dfrac{\log(ac)}{\log(b)}} + \dfrac{1}{\dfrac{\log(c)}{\log(c)}+\dfrac{\log(ab)}{\log(c)}} && 1 = \dfrac{\log(a)}{\log(a)} = \dfrac{\log(b)}{\log(b)} = \dfrac{\log(c)}{\log(c)}\\
        &= \dfrac{1}{\dfrac{\log(a)+\log(bc)}{\log(a)}} + \dfrac{1}{\dfrac{\log(b)+\log(ac)}{\log(b)}} + \dfrac{1}{\dfrac{\log(c)+\log(ab)}{\log(c)}} && \text{Add fractions}\\
        &= 1\times \dfrac{\log(a)}{\log(a)+\log(bc)} + 1\times \dfrac{\log(b)}{\log(b)+\log(ac)} + 1\times \dfrac{\log(c)}{\log(c)+\log(ab)} && \frac{1}{\frac{a}{b}} = 1\times \frac{b}{a} \\
        &= \dfrac{\log(a)}{\log(a)+\log(bc)} + \dfrac{\log(b)}{\log(b)+\log(ac)} + \dfrac{\log(c)}{\log(c)+\log(ab)} && \text{Simplify} \\
        &= \dfrac{\log(a)}{\log(abc)} + \dfrac{\log(b)}{\log(abc)} + \dfrac{\log(c)}{\log(abc)} && \text{LOL Product rule} \\
        &= \dfrac{\log(a)+\log(b)+\log(c)}{\log(abc)} && \text{Add fractions together} \\
        &= \dfrac{\log(abc)}{\log(abc)} && \text{LOL Product rule} \\
        &= 1 && \text{Simplify} \\
        & = RHS \qedhere \\
    \end{align*}
\end{proof}

\vspace{-1cm}
$$\therefore \text{since LHS = RHS, } \dfrac{1}{1+\log_a(bc)} + \dfrac{1}{1+\log_b(ac)} + \dfrac{1}{1+\log_c(ab)} = 1$$



\newpage

%% PROBLEM 5
\item \textbf{Determine} the set of real numbers $a$ for which the equation $$\log_8(x-a+1) + \log_8(x-a-1) =1$$ has \emph{exactly one} real solution for $x$.

\vspace{0.1cm}
\textbf{Solution to question 5:}

First off let's try and define the restrictions on "a". Consider the base 
log function: $log_l(m)$. "m" must be a greater than or equal to 0 since the 
log of a negative value cannot be solved. Therefore, from the question, the two 
equations being logged must not be negative:

\begin{align*}
    x - a + 1 &> 0 && \text{From } \log_8(x-a+1) \\
    x + 1 &> a && (1)\\
\end{align*}
\vspace{-1cm}
\begin{align*}
    x - a - 1 &> 0 && \text{From } \log_8(x-a-1) \\
    x - 1 &> a && (2)\\
\end{align*}

since $x+1 > x-1 > a$, we can simply say that the restriction is $x-1 > a$. 
Now let's try and convert the origianl equation into a form that is factorable 
before solving for a:

\begin{align*}
    \log_8(x-a+1) + \log_8(x-a-1) &= 1 && \text{From question} \\
    \log_8((x-a+1)(x-a-1)) &= 1 && \text{LOL Product rule, } \log_a(b) + \log_a(c) = \log_a(bc) \\
    (x-a+1)(x-a-1) &= 8^1 && \text{Convert log to exponent, } \log_a(b) = c \Longleftrightarrow a^c = b \\
    x^2-ax-x-ax+a^2+a+x-a-1 &= 8^1 && \text{Expand/ break brackets} \\
    x^2-2ax+a^2-1 &= 8 && \text{Simplify} \\
    x^2-2ax+a^2 &= 8+1 && \text{Add 1 from both sides} \\
    x^2-2ax+a^2 &= 9 && \text{Simplify} \\
    (x-a)^2 &= 3^2 && \text{Factor perfect squares} \\
    \sqrt{(x-a)^2} &= \sqrt{9} && \text{Square Root both sides} \\
    \pm (x-a) &= \pm 3 && \text{Apply } \pm \\
    x-a &= \pm 3 && \text{Equivalent to the above} \\
    x\pm 3 &= a && \text{Add a and subtract } \pm \text{3 to both sides} \\
\end{align*}

From this we now know that $a=x-3$ and $a=x+3$. 

Substituting the values of a into the 
restriction we can see that $x-3<x-1$ which simplifies to $0<2$ (subtract x, add 3) which is a true statement. 
We also find $x+3<x-1$ which simplifies to $0<-4$ (subtract x, add -3) which is a false statement. $a=x+3$ 
violates the restriction and is thus an inadmissable solution. Therefore we can confirm that $a=x-3$.
We also know that the value of a is based on x which has infinite solutions based on the 
restriction $x-1 > a$ or $x > a+1$.

$$\therefore a = x-3 \text{ where } x \in \mathbb{R}, x>a+1 $$

\newpage


\end{enumerate}
\end{document} 
