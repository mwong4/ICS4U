 %%
%%
\documentclass[12pt]{book}
\usepackage{amsfonts}
\usepackage{amsmath}
\usepackage{amssymb}
\usepackage{graphicx}
\usepackage{hyperref}
\DeclareMathOperator\cis{cis} %defining cis as an function
\setlength{\textheight}{10in}
\setlength{\textwidth}{7.4in}
\setlength{\topmargin}{-0.75in}
\setlength{\oddsidemargin}{-0.5in}
\setlength{\evensidemargin}{-0.5in}
\setlength{\parskip}{0.15in}
\setlength{\parindent}{0in}

\begin{document}


\vspace{-1.0in}\begin{center}
\Large{MHF4U :  Advanced Functions }

\Large{Assignment \#4}


\end{center}

%\medskip

\vspace{0.015in}\hrulefill\ 

\textbf{Reference Declaration} %  Fill in your Reference Declarations in this section before your submit your assignment.

Complete the Reference Declaration section below in order for your assigment to be graded.

If you used any references beyond the course text and lectures (such as other texts, discussions with peers or online resources), indicate this information in the space below.  If you did not use any aids, state this in the space provided. 

Be sure to cite appropriate theorems throughout your work. You may use shorthand for well-known theorems like the FT (Factor Theorem), RRT (Rational Root Theorem), etc. 

Note: Your submitted work must be \textbf{your original work}. 

Family Name: Wong\\%Family Name Here
First Name: Max%First Name Here

Declared References: 

% Type your references here.
% You can use as many lines as required.

\vspace{0.015in}\hrulefill\ 

\vspace{0.5cm}
\textbf{[NOTE]: Law of Logarithms is represented with LOL}

\newpage

%%%%%%%%%%%% PROBLEMS START HERE

\begin{enumerate}

%% PROBLEM 1
\item \textbf{Determine} all real numbers $x$ that satisfy the equation $\log_3(x-7) + \log_3(x-9) = 1$.

\vspace{0.5cm}
\textbf{Solve question 1 by isolating equation to one sidem performing algebra before then factoring}

\vspace{0.2cm}
First, we should identify restrictions. We know that if the parent 
logarithm is $\log_a(b) = c$, b must be positive. Taking the two "b" values 
from the equation:

\vspace{-0.2cm}
\begin{align*}
    x-7 &> 0 && \text{Subtract 7 from both sides}\\
    x &> 7 \\
\end{align*}

\vspace{-1.5cm}
\begin{align*}
    x-9 &> 0 && \text{Subtract 9 from both sides}\\
    x &> 9 \\
\end{align*}
\vspace{-0.7cm}

Since $x > 9 > 7$, we can say that the restriction is: ${x| x \in \mathbb{R}, x > 9}$
Now we can start solving for x. Start by breaking the equation down into a form 
that can be factored before factoring.

\addtolength{\jot}{1em}
\begin{align*}
    \log_3(x-7) + \log_3(x-9) &= 1 \\
    \log_3((x-7)(x-9)) &= 1 && \text{LOL Product } log_a(xy) = \log_a(x) + \log_a(y)\\
    3^1 &= (x-7)(x-9) && \log_a(b) = c \Longleftrightarrow a^c = b\\
    3 &= (x-7)(x-9) && \text{Simplify}\\
    3 &= x^2-16x+63 && \text{Multiply factors, expand}\\
    0 &= x^2-16x+63-3 && \text{Subtract 3 from both sides}\\
    0 &= x^2-16x+60 && \text{Simplify}\\
    0 &= (x-6)(x-10) && \text{Factor}\\
    x &= 6, 10 \\
\end{align*}

Now we know that $x=6, 10$ but from the restriction, x must be greater 
than 9. Therefore the solution $x=6$ is inadmissable.

$$\boxed{\therefore \text{The solution to } \log_3(x-7) + \log_3(x-9) = 1 \text{ is } x=10}$$

\newpage

%% PROBLEM 2
\item After Judgement Day, crows rose to become the dominant species and took over the planet Earth. Initially, there were 40000 crows and with no competition for food sources, this population of crows tripled in population every 1.5 years. A "mega-murder" is a murder of crows with a population of one million.

\begin{enumerate}
\item \textbf{Model} the relationship between the population of crows and time since Judgement Day.
\item \textbf{Determine} after how many years is Earth ruled by a mega-murder.
\end{enumerate}


\newpage

%% PROBLEM 3
\item \textbf{Prove} that if $a,b,c$ are the first three terms of a geometric sequence then $\log(a), \log(b), \log(c)$ are the first three terms of an arithmetic sequence.


\newpage


%% PROBLEM 4
\item Given $a,b,c \in \mathbb{R^+}$ with $a\neq1$, $b\neq1$, $c\neq1$ \textbf{prove} $$\dfrac{1}{1+\log_a(bc)} + \dfrac{1}{1+\log_b(ac)} + \dfrac{1}{1+\log_c(ab)} = 1$$.


\newpage

%% PROBLEM 5
\item \textbf{Determine} the set of real numbers $a$ for which the equation $$\log_8(x-a+1) + \log_8(x-a-1) =1$$ has \emph{exactly one} real solution for $x$.



\newpage


\end{enumerate}
\end{document} 
